\section{Truss Direct Stiffness Method}

% Direct Stiffness Method Resource:
% https://www.colorado.edu/engineering/CAS/Felippa.d/FelippaHome.d/Publications.d/Report.CU-CAS-00-13.pdf

%Assumptions: In a truss external loading is only applied to the joints and there is only axial deformation with no bending.

The \textit{Direct Stiffness Method} is a \textit{Finite Element Method} of analysis that models structural elements as springs. We will begin with deriving the axial stiffness of a structural element. \cite{Felippa2000}

\subsection{Axial Stiffness}
Members in a truss are modelled as springs with only axial deformation. The axial force in the spring is modelled by Hooke's law:

\begin{equation}
	F = k u
\end{equation}

where $F$ is the axial force, $k$ is the axial stiffness, and $u$ is the axial deformation. To define the axial stiffness we will look at a member of constant cross-secitonal area with only elastic deformation. The stress in such a member is defined by the stress-strain relationship:

\begin{equation}
	\sig = E \ep
\end{equation}

where $\sig$ is the stress, $E$ is the modulus of elasticity, and $\ep$ is the strain. Multiplying both sides of the equation by the cross-secional area, $A$, will convert the stress into a force.

\begin{align}
	\sig A &= E \ep A\\
	F & = E \ep A
\end{align}

The engineering definition of strain will be used here. Strain is defined as:

\begin{equation}
	\ep = \frac{\Delta L}{L_0} = \frac{u}{L}
\end{equation}

where $L$ is the original length of the member and $\Delta L$ is the change in length of the member. Then substituting in the definition of engineering strain into the previous equation:

\begin{align}
	F & = EA \frac{u}{L}
\end{align}

Therefore the axial stiffness from the equation is $\frac{AE}{L}$ for any structural member. The axial force of a structural element can be rewritten as:

\begin{align}
	F & = \frac{AE}{L} u
\end{align}

\subsection{Local Element Stiffness}
Each element in a structural system will have its own local element stiffness equations.

\subsection{Global Element Stiffness}


\subsection{Global Structural Stiffness}


%Boundary Conditions

