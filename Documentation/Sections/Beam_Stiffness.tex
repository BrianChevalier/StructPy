\section{Beam Direct Stiffness Method}

A beam element has four degrees of freedom. Two degrees of freedom at each node which includes the rotation $\theta$ and the deflection or transverse displacement, $w$.

\begin{figure}[h]	\centerline{\includegraphics[width=0.7\columnwidth]{Figures/DeflectionandRotation}}
	\caption{Degrees of Freedom}
	\label{fig:DeflectionandRotation}
\end{figure}

\subsection{Bending Stiffness}

We will isolate each degree of freedom by applying boundary conditions at each node, such that each beam will only have one unrestrained kinematic unknown. Figure \ref{fig:BCFrame} shows the various boundary conditions that will be used to determine the beam bending stiffness. This will allow us to derive the force-displacement relationship necessary to solve the beam.

\begin{figure}[H]	\centerline{\includegraphics[width=0.9\columnwidth]{Figures/BCFrame}}
	\caption{Beam Boundary Conditions}
	\label{fig:BCFrame}
\end{figure}

\begin{align*}
	w_0=0 & \text{	} \theta_0=?\\
	w_L=0 &\text{	}  \theta_L=0
\end{align*}

\begin{center}
	\begin{tabular}{cccc}
		$a=\frac{12EI}{L^3}$ & $b=\frac{6EI}{L^2}$ & $c=\frac{4EI}{L}$ & $d=\frac{2EI}{L}$
	\end{tabular}
\end{center}

The global element stiffness matrix for a beam is therefore:
\begin{align}
	\begin{Bmatrix}
		V_0\\ M_0\\ \hline V_L\\ M_L
	\end{Bmatrix}
	=
	\begin{bmatrix}
		a & b & -a & b\\
		b & c & -b & d\\
		-a & -b & a & -b\\
		b & d & -b & c
	\end{bmatrix}
	\begin{Bmatrix}
		w_0\\ \theta_0\\ \hline w_L\\ \theta_L
	\end{Bmatrix}
\end{align}

